\chapter{Tensor Algebra}
\section{Introduction}
To work effectively in Newtonian theory, one reallyneeds the lenguage of vectors. This lenguage, first of all, is mire succint, since it  summarized a set of three equations in one. Moreoveer, the formalism of vectors helps to solver cartain problems more readly, adn, most important of all, the language reveals structure and thereby offers insight. In exactly the same way, in relativity theory, one needs the language of tensors. Again, the language helps to summarize sets of equations succintly and to solve problems more readly, and i reveals structure in the equaions. This part is devoted to learning the formalism of tensors shich is a pre-condition for the rest.

The approach we adopt is to concentrate on the technique of tensors without taking into account the deeper geometrical significance behind the theory. We shall be concerned more with whay you do with tensors rather than what tensors actually are. There are two distinct approaches to the teaching of tensors: the abstract or index-free (coordinate-free) approach and the conventional approach based on indices. There has been a move in recent years in some quarters to introduce tensors from the stars using the more mdodern abstract approach (although some have subsequantly changed their mind and reverted to the conventional approach). The main advantage of this approach is that it offers deeper geometrical insight. However, it has two disadvantages. First of all, it requieres much more of a mathematical background, which in turn takes time to develop. The other disadvantage is that, for all its elegance, when one wants to do real calculation with tensors, as one frequently needs to, then recourse has to be made to indices. We shall adopt the more conventional index approach, because ir will prove faster and more practical. However, we advise those who wish to take their study of the subject further to look at the index-free approach at the first oppoertunity.
